\newcommand{\re}{{\mathbb{R}}}
\newcommand{\ve}[1]{\mbox{\boldmath $#1$}}
\newcommand{\ba}{\ve{a}}
\newcommand{\bb}{\ve{b}}
\newcommand{\bA}{\ve{A}}
\newcommand{\bu}{\ve{u}}
\newcommand{\bv}{\ve{v}}
\newcommand{\br}{\ve{r}}
\newcommand{\bt}{\ve{t}}
\newcommand{\bn}{\ve{n}}
\newcommand{\bc}{\ve{c}}
\newcommand{\bff}{\ve{f}}
\newcommand{\bx}{\ve{x}}
\newcommand{\hx}{\hat{x}}
\newcommand{\hbx}{\hat{\ve{x}}}
\newcommand{\bS}{\ve{S}}
\newcommand{\bs}{\ve{s}}
\newcommand{\bU}{\ve{U}}
\newcommand{\bV}{\ve{V}}
\newcommand{\bQ}{\ve{Q}}
\newcommand{\by}{\ve{y}}
\newcommand{\be}{\ve{e}}
\newcommand{\bP}{\ve{P}}
\newcommand{\bGG}{\ve{G}}
\newcommand{\bRR}{\ve{R}}
\newcommand{\bq}{\ve{q}}
\newcommand{\bLambda}{\ve{\boldsymbol{\Lambda}}}
\newcommand{\bSigma}{\ve{\boldsymbol{\Sigma}}}
\newcommand{\bdelta}{\ve{\boldsymbol{\delta}}}
\newcommand{\beps}{\ve{\boldsymbol{\varepsilon}}}
\newcommand{\nlat}{\mathit{nlat}}
\newcommand{\nlon}{\mathit{nlon}}


\documentclass[11pt]{article}


\usepackage{color}
\usepackage{pifont}
%%% PAGE DIMENSIONS

\usepackage{enumerate}
\usepackage{ulem}

%%% PACKAGES
\usepackage{booktabs} 
\usepackage{array} 
\usepackage{paralist} 
\usepackage{verbatim} 
\usepackage{tabu}
\usepackage{epsf}
\usepackage{amsmath}
\usepackage{amssymb}
\usepackage{ulem}
\usepackage{stmaryrd}
\usepackage{setspace}
\usepackage{enumerate}
\usepackage{graphicx}
\usepackage{fancyhdr}
\usepackage{float}
\usepackage{transparent}
\usepackage{sectsty}
\usepackage{color}
\usepackage{rotate} 
\usepackage{breqn}
   

\fancypagestyle{plain}{\fancyfoot[C]{\vspace*{1.5\baselineskip}\thepage}}

\textwidth = 6.5 in
\textheight = 8 in
\oddsidemargin = 0.0 in
\evensidemargin = 0.0 in
\topmargin = 0.0 in
\headheight = 0.0 in
\headsep = 0.0 in
\parskip = 0.0in
\parindent = 0.0in


\begin{document}

\begin{center}
	{\Large Spherical harmonics and {\texttt{SPHEREPACK}}}\\
	\smallskip
	Spencer H. Bryngelson
\end{center}

Capsules in the three-dimensional cell code are represented using spherical harmonics,
as compared to Fourier harmonics in the two-dimensional analog. \\

Per the {\texttt{SPHEREPACK}} documentation: 
\begin{quote}
	Scalar and vector harmonic analyses are used for problem solving in spherical coordinates 
	in the same way that Fourier analysis is used in Cartesian coordinates. Thus the package 
	provides the same high level of accuracy that the spectral method provides in Cartesian 
	coordinates. Furthermore, accuracy is uniform on the sphere and independent of the location 
	of the poles. This eliminates a number of computational difficulties associated with solving partial 
	differential equations on the sphere including the accuracy and stability problems that can be created 
	by the clustering of points near the pole. Interpolation and smoothing on the sphere are also 
	significantly facilitated by harmonic analysis (Swarztrauber, 1979). Indeed the package includes 
	programs for interpolating between Gaussian and equally-spaced latitudinal grids.
\end{quote}

The spherical harmonics are a complete set of orthogonal functions on the sphere, 
and thus may be used to represent functions defined on the surface of a sphere, just as 
circular functions (sines and cosines) are used to represent functions on a circle via Fourier series. \\

The notation used for \texttt{SPHEREPACK} routines follows: 
\begin{itemize}
	\item \texttt{sh} or \texttt{vh} denote scalar and vector harmonic transforms respectively. 
	\item \texttt{a} or \texttt{s} denotes analysis (physical to wave space) or synthesis
 (wave to physical space). 
	\item \texttt{g} or \texttt{e} denotes Gaussian or equally-spaced latitudinal grid.
 	\item \texttt{c} or \texttt{s} denotes whether or not quantities are computed or stored.
\end{itemize}

In practice, scalar spherical harmonic analyses are used to transform the 
capsule surface(s) from physical to wave space where the derivatives are computed
for the constitutive model, then syntheses are used to transform back to physical space and 
advect the surface(s) according to $\text{d}_t \bx = \bu(\bx)$. 

\subsection*{Associated Legendgre polynomials and Gaussian weights}

Here $\bar{P}_n^m$ are the associated Legendre functions defined by
\begin{gather}
	\bar{P}_n^m(\theta) = \sqrt{ \frac{2n+1}{2} \frac{ (n-m)! }{ (n+m)! } } \frac{\sin^m \theta }{ 2^n \, n!}
	\frac{ \text{d}^{n+m} (x^2 -1)^n }{\text{d}x^{n+m}} ,
\end{gather}
where $x = \cos \theta$, and $\theta$ and $w$ are the Gaussian points and weights on the sphere,
\begin{align*}
	\text{Gaussian points in radians:} \qquad &\theta(1), \theta(2), \ldots, \theta(\nlat), \\
	\text{Corresponding Gaussian weights:} \qquad &w(1), w(2), \ldots, w(\nlat).
\end{align*}

\subsection*{\texttt{SHAGS}: Scalar spherical harmonic {\color{red}analysis} on a Gaussian latitudinal grid}

This routine is repeatedly used to transform from physical to wave space on a Gaussian grid.
The input is the physical capsule collocation points $\bx$ and the output are the arrays of spherical
harmonic coefficients $\ba$ and $\bb$. Associated Legendre polynomials are stored rather
than recomputed. \\

The physical points $\bx$ are arranged in an array of dimension $\nlat \times \nlon$ and
 the coefficient arrays $\ba$ and $\bb$ are of dimension $\nlat \times \nlat$, where $\nlat$ is the number of 
lateral modes, so long as the number of longitudinal modes $\nlon = 2 \, \nlat$ is even. They are defined 
element-wise by
\begin{gather}
	a(m+1,n+1) = \sum_{i=1}^{\nlat} c(m+1,i) \, w(i) \, \bar{P}_n^m(\theta[i]) 
	\quad \text{for} \quad m=0,\nlat-1; \; n=m,\nlat-1, \\
	b(m+1,n+1) = \sum_{i=1}^{\nlat} s(m+1,i) \, w(i) \, \bar{P}_n^m(\theta[i]) 
	\quad \text{for} \quad m=0,\nlat-1; \; n=m,\nlat-1,
\end{gather}
where $c$ and $s$ are the cosine and sine part of the Fourier transform of $\bx$, 
\begin{align}
	c(m,i) &= \frac{2}{\nlon} \sum_{j=1}^\nlon \frac{x(i,j)}{k(j)} \cos\left[ \frac{2 \pi (m-1)(j-1)}{\nlon} \right] 
	\quad \text{where} \quad
	k(j) = \begin{cases} 2 & j=1 \, \text{or} \, \nlon \\ 1 & \text{otherwise,} \end{cases}\\
	s(m,i) &= \frac{2}{\nlon} \sum_{j=2}^\nlon x(i,j) \sin\left[ \frac{2 \pi (m-1)(j-1)}{\nlon} \right].
\end{align}
Note that $\ba$ and $\bb$ are triangular, and thus only $\nlat^2$ unique harmonic 
coefficients exist for $\nlat \times \nlon = 2 \times \nlat^2$ collocation points. 

\subsection*{\texttt{SHSGS}: Scalar spherical harmonic {\color{red}synthesis} on a Gaussian latitudinal grid}

This routine is used to transform from spherical harmonic coefficients $\ba$ and $\bb$ to 
physical points $x(i,j)$.  The physical points are given by
\begin{multline}
	x(i,j) = \sum_{n=0}^{\nlat-1} \frac{1}{2} \, \bar{P}_{n}^0(\theta[i]) \, a(1,n+1) + \\
		\sum_{m=1}^{\nlat-1} \sum_{n=m}^{\nlat-1} \bar{P}_{n}^m(\theta[i])  \left(a(m+1,n+1) \cos\left( m \, \phi[j] \right)
		- b(m+1,n+1) \sin\left( m \, \phi[j] \right) \right),
\end{multline}
which are the Gaussian colatitudinal points $\theta(i)$ and longitudinal points $\phi(j) = 2 \pi (j-1) / \nlon$.
	
	
\subsection*{Basics}

\begin{gather}
	f_{i,j} = f(\theta_i,\phi_j)  = \sum_{n=0}^N \sum_{m=-n}^n c_{m,n} Y_n^m (\theta_i,\phi_j) 
\end{gather}

\begin{align}
	Y_n^m (\theta_i, \phi_j) &= \bar{P}_n^m (\theta_i) e^{i m \phi_j} \\
	&= \sqrt{ \frac{2n+1}{4\pi} \frac{(n-m)!}{(n+m)!} } \bar{P}_n^m (\cos \theta_i) e^{i m \phi_j}
\end{align}


Least squares harmonic analysis: On a grid with $2N$ longitudes and $N$ latitudes
the discrete harmonic basis consists of the functions
\begin{gather}
	\cos m \phi \bar{P}_n^m(\theta) \quad \text{and} \quad \sin m \phi \bar{P}_n^m (\theta)
\end{gather}
for $m \le n$ and $n \le N$, which yields a total of $N^2$ basis functions. However,
this is half the number of grid points $2N^2$, which implies the interpolation problem
on the sphere does not have a solution.

\subsection*{Orthogonality}

The associated Legendre polynomials are orthogonal on $[-1,1]$ with weighting function 1,
\begin{gather}
	\int_{-1}^1 \bar{P}_n^m (x) \bar{P}_{n'}^m (x) \, \text{d}x = 
	\frac{2}{2n+1} \frac{ (n+m)! }{ (n-m)! }  \, \delta_{n \, n'},
\end{gather}
and orthogonal on $[-1,1]$ w.r.t $m$ with weight function $(1-x^2)^{-1}$,
\begin{gather}
	\int_{-1}^1 \bar{P}_n^m (x) \bar{P}_{n'}^m (x) \, \frac{\text{d}x}{1-x^2} = 
	\frac{1}{m} \frac{ (n+m)! }{ (n-m)! } \, \delta_{m \, m'}.
\end{gather}
The spherical harmonics $Y_n^m$ are normalized such that
\begin{align*}
	\int_{0}^{2\pi} &\int_{0}^\pi Y_n^m (\theta,\phi) \bar{Y}_{n'}^{m'} (\theta,\phi) \, \sin\theta \, \text{d}\theta \text{d}\phi \\
	&= \int_0^{2\pi} \int_{-1}^1  Y_n^m (\theta,\phi) \bar{Y}_{n'}^{m'} (\theta,\phi) \, \text{d}(\cos \theta) \text{d} \phi \\
	&= \delta_{m \, m'} \delta_{n \, n'},
\end{align*}
where the bar denotes the complex conjugate. \\

Define:\\

$x(i,j)$: a two or three dimensional array (see input parameter
        $nt$) that contains the discrete function to be analyzed.
            $x(i,j)$ contains the value of the function at the gaussian
            point $\theta(i)$ and longitude point $\phi(j) = (j-1)2\pi/\nlon$
            the index ranges are defined above at the input parameter
            $isym$.
            
\subsection*{Manufactured test solution}
Linearization of the dynamics of the full system proceeds as:
\begin{gather}
	u(\bx + \bdelta) = \frac{ \partial (\bx + \bdelta) }{ \partial t } = \frac{ \partial \bx }{ \partial t } + A \bdelta + \mathcal{O}(\bdelta^2)
\label{e:full}
\end{gather}
which has the nonlinear manufactured solution
\begin{gather}
	u(\bx) = \bx + \bx^3.
\end{gather}
So then column $i$ of $A$ is simply
\begin{gather}
	A_i = \frac{ u(\bx+\bdelta_i) - u(\bx)}{ \delta_i },
\end{gather}
where $\delta_i = \lvert \bdelta_i \rvert$, which for the given manufactured solution
should recover
\begin{gather}
	A_i = \frac{ (1+3\bx^2)\bdelta_i + 3 \bx \bdelta_i^2+ \bdelta_i^3 }{ \delta_i } 
	      \approx \frac{ (1+3\bx^2)\bdelta_i }{ \delta_i } = c_x \frac{ \bdelta_i}{ \delta_i }
\end{gather}	
for sufficiently small $\delta_i$, where $c_x$ is a specific constant that 
depends only on the base configuration $\bx$. Similarly, for a case with 
$\bdelta_{ij} = \bdelta_i + \bdelta_j$,
which was not used in the construction of $A$,
\begin{align*}
	u(\bx + \bdelta_{ij}) &= \bx + \bdelta_{ij} + \bx^3 + 3 \bx^2 \bdelta_{ij}  + 3 \bdelta_{ij}^2 \bx + \bdelta_{ij}^3, \\
	N(\bdelta_{ij}) = u(\bx + \bdelta_{ij}) - u(\bx) &= c_x \bdelta_{ij} + 3 \bx \bdelta_{ij}^2 + \bdelta_{ij}^3,
\end{align*}
which is the exact nonlinear solution. The linear approximation due to $A$ is given by
\begin{gather}
	\frac{\partial \beps}{\partial t} = A \beps + \mathcal{O}(\beps^2),
\end{gather}
where $\beps = \bdelta_{ij}$ in this case.  Then the velocity due to $\bdelta_{ij}$ 
according to the linearization is
\begin{align*}
	L(\bdelta_{ij}) &= A_i \bdelta_i + A_j \bdelta_j \\
	&= c_x (\bdelta_i + \bdelta_j)
\end{align*}
and the linearization error is
\begin{align*}
	E(\bdelta_{ij}) &= N(\bdelta_{ij}) - L(\bdelta_{ij}) \\
		&= 3 \bx \bdelta_{ij}^2 + \bdelta_{ij}^3 \\
		&= \mathcal{O}(\bdelta_{ij}^2)
\end{align*} 
	
Similarly one can recast collocation points $\bx$ in terms of spherical harmonic coefficients
$\ba$, in which case (\ref{e:full}) can be rewritten as
\begin{gather}
	u(\ba + \bdelta) = \frac{ \partial (\ba + \bdelta) }{ \partial t } = \frac{ \partial \ba }{ \partial t } + A \bdelta + \mathcal{O}(\bdelta^2),
\label{e:fulla}
\end{gather}
where $A$ is distinct from the $A$ in (\ref{e:full}) but contains the same information 
about the linearization of the dynamics.  In this case, $\bdelta_i$ is interpreted
as disturbances of spherical mode coefficients, and thus periodic spherical modes on the
unit sphere.  The linearization process is identical, but the calculation of $u( \cdotp )$ is not.
The procedure is:
\begin{enumerate}
	\item Transform spherical mode coefficients to physical points $\ba \to \bx$
	\item Compute $u(\bx) = \bx + \bx^3$
	\item Reverse transform $u(\bx) \to \ba + \ba^3 = u(\ba)$
\end{enumerate}
at which point the same linearization error holds.





\end{document}